% !TeX TXS-program:compile = txs:///arara
% arara: pdflatex: {shell: yes, synctex: no, interaction: batchmode}
% arara: pdflatex: {shell: yes, synctex: no, interaction: batchmode} if found('log', '(undefined references|Please rerun|Rerun to get)')

\documentclass{article}
\usepackage[french]{babel}
\usepackage[utf8]{inputenc}
\usepackage[T1]{fontenc}
\usepackage[fr]{Scrabble}
\usepackage[upright]{fourier}
\usepackage[scaled=0.875]{helvet}
\renewcommand\ttdefault{lmtt}
\usepackage[scaled=0.875]{cabin}
\usepackage{amsmath,amssymb}
\usepackage{fontawesome5}
\usepackage{enumitem}
\usepackage{tabularray}
\usepackage{fancyvrb}
\usepackage{fancyhdr}
\fancyhf{}
\renewcommand{\headrulewidth}{0pt}
\lfoot{\sffamily\small [Scrabble]}
\cfoot{\sffamily\small - \thepage{} -}
\rfoot{\hyperlink{matoc}{\small\faArrowAltCircleUp[regular]}}

\usepackage{hvlogos}
\usepackage{hyperref}
\urlstyle{same}
\hypersetup{pdfborder=0 0 0}
\usepackage[margin=1.5cm]{geometry}
\setlength{\parindent}{0pt}
\definecolor{LightGray}{gray}{0.9}

\def\TPversion{0.1.1}
\def\TPdate{6 Janvier 2023}

\usepackage[most]{tcolorbox}
\tcbuselibrary{minted}
\NewTCBListing{PresentationCode}{ O{blue} m }{%
	sharp corners=downhill,enhanced,arc=12pt,skin=bicolor,%
	colback=#1!1!white,colframe=#1!75!black,colbacklower=white,%
	attach boxed title to top right={yshift=-\tcboxedtitleheight},title=Code \LaTeX,%
	boxed title style={%
		colframe=#1!75!black,colback=#1!15!white,%
		,sharp corners=downhill,arc=12pt,%
	},%
	fonttitle=\color{#1!90!black}\itshape\ttfamily\footnotesize,%
	listing engine=minted,minted style=colorful,
	minted language=tex,minted options={tabsize=4,fontsize=\footnotesize,autogobble},
	#2
}

\newcommand\Cle[1]{{\bfseries\sffamily\textlangle #1\textrangle}}

\begin{document}

\pagestyle{fancy}

\thispagestyle{empty}

\vspace{2cm}

\begin{center}
	\begin{minipage}{0.75\linewidth}
	\begin{tcolorbox}[colframe=yellow,colback=yellow!15]
		\begin{center}
			\begin{tabular}{c}
				{\Huge \texttt{Scrabble [fr]}}\\
				\\
				{\LARGE Un plateau de Scrabble,} \\
				\\
				{\LARGE avec mots ou non.} \\
				\\
				{Scrabble\texttrademark{} est une marque déposée de Hasbro\texttrademark{} et Mattel\texttrademark{}.}
			\end{tabular}
			
			\medskip
			
			{\small \texttt{Version \TPversion{} -- \TPdate}}
		\end{center}
	\end{tcolorbox}
\end{minipage}
\end{center}

\vspace{0.5cm}

\begin{center}
	\begin{tabular}{c}
	\texttt{Cédric Pierquet}\\
	{\ttfamily c pierquet -- at -- outlook . fr}\\
	\texttt{\url{https://github.com/cpierquet/Scrabble}}
\end{tabular}
\end{center}

\vspace{0.5cm}

{$\blacktriangleright$~~Quelques commandes pour afficher un plateau de Scrabble, avec ou sans mots.}

\smallskip

{$\blacktriangleright$~~Possibilité de mettre le plateau et les points pour les versions Française, Anglaise, Allemande et Espagnole.}

\smallskip

{$\blacktriangleright$~~Idée(s) venant de \url{https://tex.stackexchange.com/questions/194780/tikz-drawing-a-rectangle-with-spikes-on-borders}}

\vspace{0.5cm}

\begin{center}
	\PlateauScrabble[Echelle=0.5]
	~~~~
	\begin{EnvScrabble}[Echelle=0.5]
		\ScrabblePlaceMot{T,I,K,Z}{1,13}
		\ScrabblePlaceMot[V]{P,S,T,R,I,C,K,S}{1,15}
		\ScrabblePlaceMot[V]{K,A,T,E,X}{3,13}
		\ScrabblePlaceMot{C,L,E,V,E,R,E,F}{1,10}
		\ScrabblePlaceMot[V]{M,E,T,A,P,O,S,T}{7,11}
		\ScrabblePlaceMot{L,U,A,L,A,T,E,X}{5,8}
		\ScrabblePlaceMot[V]{P,R,O,F,L,Y,*,E,E}{11,15}
		\ScrabblePlaceMot{P,R,O,F,C,O,L,L,E,*,E}{1,1}
		\ScrabblePlaceMot{S,Y,N,T,A,X,E}{7,5}
		\ScrabblePlaceMot[V]{S,T,R,I,N,*}{10,6}
	\end{EnvScrabble}
\end{center}

\vspace{0.5cm}

\hfill{}\textit{Merci aux membres du groupe \faFacebook{} du \og Coin \LaTeX{} \fg{} pour leur aide et leurs idées !}

\vfill

\hrule

\medskip

\begin{tblr}{width=\linewidth,colspec={X[c]X[c]X[c]X[c]X[c]X[c]},cells={font=\sffamily}}
{\huge \LaTeX} & & & & &\\
& {\huge \pdfLaTeX} & & & & \\
& & {\huge \LuaLaTeX} & & & \\
& & & {\huge \TikZ} & & \\
& & & & {\huge \TeXLive} & \\
& & & & & {\huge \MiKTeX} \\
\end{tblr}

\medskip

\hrule

\medskip

\newpage

\part*{Historique}

\verb|v0.1.1|~:~~~~Option \textsf{[fr]} pour franciser les commandes

\verb|v0.1  |~:~~~~Version initiale

\part*{Introduction}

\section{Le package Scrabble}

\subsection{Source}

Certaines idées viennent de \url{https://tex.stackexchange.com/questions/194780/tikz-drawing-a-rectangle-with-spikes-on-borders}, avec une proposition de Mark Wibrow.

\smallskip

Le package a ensuite été \textit{construit} et \textit{modestement enrichi} autour de styles et méthodes proposées par Mark Wibrow.

\subsection{Chargement du package, packages utilisés}

Le package \textsf{Scrabble} se charge dans le préambule via la commande :

\begin{PresentationCode}{listing only}
\usepackage[fr]{Scrabble}
\end{PresentationCode}

Il est compatible avec les compilations usuelles en \textsf{latex}, \textsf{pdflatex}, \textsf{lualatex} ou \textsf{xelatex}.

\medskip

Il charge les packages et librairies suivantes :

\begin{itemize}
	\item \texttt{tikz} avec les librairies \Cle{calc} et \Cle{shapes.geometric} ;
	\item \texttt{pgf} et \texttt{pgffor} ;
	\item \texttt{xstring} ;
	\item \texttt{xparse} ;
	\item \texttt{simplekv}.
\end{itemize}

\subsection{\og Philosophie \fg{} du package}

L'idée est de proposer, grâce à des commandes en \TikZ, des commandes ou environnements pour présenter un plateau de Scrabble\texttrademark{} :

\begin{itemize}
	\item \textit{autonome} ;
	\item dans un \textit{environnement} avec rajout éventuel de mots.
\end{itemize}

\begin{PresentationCode}{listing only}
%commande autonome pour afficher le plateau (vide)
\PlateauScrabble<langue>[clés]

%environnement avec mot(s)
\begin{EnvScrabble}<langue>[clés]
	\ScrabblePlaceMot[orientation]{l,e,t,t,r,e,s,}{coordonnées de la case de départ}
\end{EnvScrabble}
\end{PresentationCode}

\subsection{Langues}

Les \textsf{commandes}, \textsf{environnements} et \textsf{clés} proposées le sont en version \textsf{française}, mais les cases et points peuvent être affichées en :

\begin{itemize}
	\item français ;
	\item anglais ;
	\item allemand ;
	\item espagnol.
\end{itemize}

\pagebreak

\section{Commandes, clés et options}

\subsection{Le plateau seul}

Le premier argument, \textit{optionnel}, entre \texttt{<...>} est la \Cle{langue} d'affichage peut-être choisie parmi :

\hfill\Cle{FR} (français, par défaut), \Cle{EN} (anglais), \Cle{DE} (allemand) et \Cle{ES} (espagnol).\hfill~

\smallskip

Le second argument, \textit{optionnel}, entre \texttt{[...]} propose les \Cle{clés} :

\begin{itemize}
	\item \Cle{Echelle} pour spécifier l'échelle d'affichage (l'unité de base étant 1~cm) ; \hfill~défaut : \Cle{1}
	\item le booléen \Cle{Cadre} pour afficher un cadre autour du plateau ; \hfill~défaut : \Cle{true}
	\item le booléen \Cle{Labels} pour afficher le \textit{nom} des cases ;\hfill~défaut : \Cle{true}
	\item le booléen \Cle{Aide} pour afficher une aide pour repérer les cases ;\hfill~défaut : \Cle{false}
\end{itemize}

\begin{PresentationCode}{}
%plateau sans le nom des cases
\PlateauScrabble[Labels=false,Echelle=0.55]\\
%plateau en anglais
\PlateauScrabble<EN>[Echelle=0.55,Cadre=false]
\end{PresentationCode}

\newpage

\subsection{Le plateau avec des mots}

Il s'agit dans ce cas d'utiliser l'\textsf{environnement} ainsi que la \textsf{commande} spécifique pour placer les mots.

\smallskip

En ce qui concerne les options de l'\textsf{environnement}, ce sont les mêmes que pour la \textsf{commande} autonome.

\smallskip

Pour le placement des mots :

\begin{itemize}
	\item le premier argument, \textit{optionnel}, entre \texttt{[...]} est l'orientation du mot, à choisir entre \Cle{H} (par défaut) et \Cle{V} (en fait toute autre lettre que \Cle{H} !) ;
	\item le deuxième argument, \textit{mandataire}, entre \texttt{\{...\}}, est la liste des lettres à placer, séparées par des \og \texttt{,} \fg{} ;
	\item le dernier argument, \textit{mandataire}, entre \texttt{\{...\}}, correspond aux coordonnées de la case sur laquelle sera placée le début du mot (la case (1;\,1) étant la case au bord Bas/Gauche).
\end{itemize}

\textbf{Remarque 1 :} la langue choisie permettra également d'avoir les jetons avec le \textit{bon} nombre de points !

\smallskip

\textbf{Remarque 2 :} le \textit{blanc} (ou \textit{joker}) est obtenu par la lettre \texttt{*}.

\begin{PresentationCode}{}
\begin{EnvScrabble}[Echelle=0.75,Labels=false,Aide]
	\ScrabblePlaceMot{T,I,K,Z}{1,13}
	\ScrabblePlaceMot[V]{P,S,T,R,I,C,K,S}{1,15}
	\ScrabblePlaceMot[V]{K,A,T,E,X}{3,13}
	\ScrabblePlaceMot{C,L,E,V,E,R,E,F}{1,10}
	\ScrabblePlaceMot[V]{M,E,T,A,P,O,S,T}{7,11}
	\ScrabblePlaceMot{L,U,A,L,A,T,E,X}{5,8}
	\ScrabblePlaceMot[V]{P,R,O,F,L,Y,*,E,E}{11,15}
	\ScrabblePlaceMot{P,R,O,F,C,O,L,L,E,*,E}{1,1}
	\ScrabblePlaceMot{S,Y,N,T,A,X,E}{7,5}
	\ScrabblePlaceMot[V]{S,T,R,I,N,*}{10,6}
\end{EnvScrabble}
\end{PresentationCode}

\begin{PresentationCode}{}
\begin{EnvScrabble}<FR>[Echelle=0.55]
	\ScrabblePlaceMot{T,I,K,Z}{1,13}
	\ScrabblePlaceMot[V]{P,S,T,R,I,C,K,S}{1,15}
	\draw (7,15) node[font=\LARGE\sffamily] {Version française} ; %code rajouté
\end{EnvScrabble}~~~
\begin{EnvScrabble}<EN>[Echelle=0.55]
	\ScrabblePlaceMot{T,I,K,Z}{1,13}
	\ScrabblePlaceMot[V]{P,S,T,R,I,C,K,S}{1,15}
	\draw (7,15) node[font=\LARGE\sffamily] {Version anglaise} ; %code rajouté
\end{EnvScrabble}\\
\begin{EnvScrabble}<DE>[Echelle=0.55]
	\ScrabblePlaceMot{T,I,K,Z}{1,13}
	\ScrabblePlaceMot[V]{P,S,T,R,I,C,K,S}{1,15}
	\draw (7,15) node[font=\LARGE\sffamily] {Version allemande} ; %code rajouté
\end{EnvScrabble}~~~
\begin{EnvScrabble}<ES>[Echelle=0.55]
	\ScrabblePlaceMot{L,A,T,E,X}{1,14}
	\ScrabblePlaceMot[V]{M,A,D,R,I,D}{2,15}
	\draw (7,15) node[font=\LARGE\sffamily] {Version espagnole} ; %code rajouté
\end{EnvScrabble}
\end{PresentationCode}

\end{document}