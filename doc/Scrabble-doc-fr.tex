% !TeX TXS-program:compile = txs:///arara
% arara: pdflatex: {shell: yes, synctex: no, interaction: batchmode}
% arara: pdflatex: {shell: yes, synctex: no, interaction: batchmode} if found('log', '(undefined references|Please rerun|Rerun to get)')

\documentclass{article}
\usepackage[french]{babel}
\usepackage[utf8]{inputenc}
\usepackage[T1]{fontenc}
\usepackage{Scrabble}
%\usepackage[upright]{fourier}
%\usepackage[scaled=0.875]{helvet}
%\renewcommand\ttdefault{lmtt}
%\usepackage{cabin}
\usepackage{amsmath,amssymb}
\usepackage{fontawesome5}
\usepackage{enumitem}
\usepackage{tabularray}
\usepackage{fancyvrb}
\usepackage{fancyhdr}
\fancyhf{}
\renewcommand{\headrulewidth}{0pt}
\lfoot{\sffamily\small [Scrabble]}
\cfoot{\sffamily\small - \thepage{} -}
\rfoot{\hyperlink{matoc}{\small\faArrowAltCircleUp[regular]}}

%\usepackage{hvlogos}
\usepackage{hologo}
\providecommand\tikzlogo{Ti\textit{k}Z}
\providecommand\TeXLive{\TeX{}Live\xspace}
\providecommand\PSTricks{\textsf{PSTricks}\xspace}
\let\pstricks\PSTricks
\let\TikZ\tikzlogo
\newcommand\TableauDocumentation{%
	\begin{tblr}{width=\linewidth,colspec={X[c]X[c]X[c]X[c]X[c]X[c]},cells={font=\sffamily}}
		{\huge \LaTeX} & & & & &\\
		& {\huge \hologo{pdfLaTeX}} & & & & \\
		& & {\huge \hologo{LuaLaTeX}} & & & \\
		& & & {\huge \TikZ} & & \\
		& & & & {\huge \TeXLive} & \\
		& & & & & {\huge \hologo{MiKTeX}} \\
	\end{tblr}
}

\usepackage{hyperref}
\urlstyle{same}
\hypersetup{pdfborder=0 0 0}
\usepackage[margin=1.5cm]{geometry}
\setlength{\parindent}{0pt}
\definecolor{LightGray}{gray}{0.9}

\def\TPversion{0.1.6}
\def\TPdate{26 janvier 2024}

\usepackage[most]{tcolorbox}
\tcbuselibrary{minted}
\NewTCBListing{PresentationCode}{ O{blue} m }{%
	sharp corners=downhill,enhanced,arc=12pt,skin=bicolor,%
	colback=#1!1!white,colframe=#1!75!black,colbacklower=white,%
	attach boxed title to top right={yshift=-\tcboxedtitleheight},title=Code \LaTeX,%
	boxed title style={%
		colframe=#1!75!black,colback=#1!15!white,%
		,sharp corners=downhill,arc=12pt,%
	},%
	fonttitle=\color{#1!90!black}\itshape\ttfamily\footnotesize,%
	listing engine=minted,minted style=colorful,
	minted language=tex,minted options={tabsize=4,fontsize=\footnotesize,autogobble},
	#2
}

\newcommand\Cle[1]{{\bfseries\sffamily\textlangle #1\textrangle}}

\begin{document}

\pagestyle{fancy}

\thispagestyle{empty}

\vspace{2cm}

\begin{center}
	\begin{minipage}{0.75\linewidth}
	\begin{tcolorbox}[colframe=yellow,colback=yellow!15]
		\begin{center}
			\begin{tabular}{c}
				{\Huge \texttt{Scrabble [fr]}}\\
				\\
				{\LARGE Un plateau de Scrabble,} \\
				\\
				{\LARGE avec mots ou non.} \\
				\\
				{Scrabble\texttrademark{} est une marque déposée de Hasbro\texttrademark{} et Mattel\texttrademark{}.}
			\end{tabular}
			
			\medskip
			
			{\small \texttt{Version \TPversion{} -- \TPdate}}
		\end{center}
	\end{tcolorbox}
\end{minipage}
\end{center}

\vspace{0.5cm}

\begin{center}
	\begin{tabular}{c}
	\texttt{Cédric Pierquet}\\
	{\ttfamily c pierquet -- at -- outlook . fr}\\
	\texttt{\url{https://github.com/cpierquet/Scrabble}}
\end{tabular}
\end{center}

\vspace{0.5cm}

{$\blacktriangleright$~~Quelques commandes pour afficher un plateau de Scrabble, avec ou sans mots.}

\smallskip

{$\blacktriangleright$~~Possibilité de mettre le plateau et les points pour les versions Française, Anglaise, Allemande et Espagnole.}

\smallskip

{$\blacktriangleright$~~Possibilité d'insérer un mot en mode \textit{en ligne}.}

\smallskip

{$\blacktriangleright$~~Idée(s) venant de \scalebox{0.8}[1]{\ttfamily\url{ https://tex.stackexchange.com/questions/194780/tikz-drawing-a-rectangle-with-spikes-on-borders}}}
%
%{\centering \small \url{https://tex.stackexchange.com/questions/194780/tikz-drawing-a-rectangle-with-spikes-on-borders}}

\vspace{0.15cm}

\begin{center}
	\PlateauScrabble[Echelle=0.5]
	~~~~
	\begin{EnvScrabbleFR}[Echelle=0.5]
		\ScrabblePlaceMot{TIKZ}{1,13}
		\ScrabblePlaceMot[V]{pstricks}{1,15}
		\ScrabblePlaceMot[V]{KaTeX}{3,13}
		\ScrabblePlaceMot{cleveref}{1,10}
		\ScrabblePlaceMot[V]{METAPOST}{7,11}
		\ScrabblePlaceMot{LUALATEX}{5,8}
		\ScrabblePlaceMot[V]{ProfLy*ee}{11,15}
		\ScrabblePlaceMot{PROFCOLLE*E}{1,1}
		\ScrabblePlaceMot{SYNTAXE}{7,5}
		\ScrabblePlaceMot[V]{STRIN*}{10,6}
	\end{EnvScrabbleFR}
\end{center}

\vspace{0.5cm}

%\hfill{}\textit{Merci aux membres du groupe \faFacebook{} du \og Coin \LaTeX{} \fg{} pour leur aide et leurs idées !}

\hfill{}\textit{Merci à Denis Bitouzé, à Patrick Bideault, à quark67 et à jmbeckers pour leurs retours et idées !}

\vfill

\hrule

\medskip

\TableauDocumentation

\medskip

\hrule

\medskip

\newpage

\part*{Introduction}

\section{Le package Scrabble}

\subsection{Source}

Certaines idées viennent de \scalebox{0.75}[1]{\ttfamily\url{ https://tex.stackexchange.com/questions/194780/tikz-drawing-a-rectangle-with-spikes-on-borders}}, avec une proposition de Mark Wibrow.

\smallskip

Le package a ensuite été \textit{construit} et \textit{modestement enrichi} autour de styles et méthodes proposées par Mark Wibrow.

\subsection{Chargement du package, packages utilisés}

Le package \textsf{Scrabble} se charge dans le préambule via la commande :

\begin{PresentationCode}{listing only}
\usepackage{Scrabble}
\end{PresentationCode}

Il est compatible avec les compilations usuelles en \textsf{latex}, \textsf{pdflatex}, \textsf{lualatex} ou \textsf{xelatex}.

\medskip

Il charge les packages et librairies suivantes :

\begin{itemize}
	\item \texttt{tikz} avec les librairies \Cle{calc} et \Cle{shapes.geometric} ;
	\item \texttt{pgf} et \texttt{pgffor} ;
	\item \texttt{xstring}, \texttt{xparse}, \texttt{simplekv} et \texttt{listofitems}.
\end{itemize}

\subsection{\og Philosophie \fg{} du package}

L'idée est de proposer, grâce à des commandes en \TikZ, des \textsf{commandes} ou \textsf{environnements} pour présenter un plateau de Scrabble\texttrademark{} :

\begin{itemize}
	\item \textit{autonome} ;
	\item dans un \textit{environnement} avec rajout éventuel de mots.
\end{itemize}

\begin{PresentationCode}{listing only}
%commande autonome pour afficher le plateau (vide)
\PlateauScrabble<langue>[clés]

%environnement francisé, avec clés en français, avec mot(s)
\begin{EnvScrabbleFR}<langue>[clés]
	\ScrabblePlaceMot[orientation]{mot}{coordonnées de la case de départ}
	\ScrabblePlaceMot[orientation]{mot}{coordonnées de la case de départ}
\end{EnvScrabbleFR}
\end{PresentationCode}

\subsection{Langues}

Les \textsf{commandes}, \textsf{environnements} et \textsf{clés} proposées le sont en version \textsf{française} (et \textsf{anglaise}, voir fin de la doc), et les cases et points peuvent être affichées en :

\begin{itemize}
	\item français (code ISO 639-1 FR) ;
	\item anglais (code ISO 639-1 EN) ;
	\item allemand (code ISO 639-1 DE) ;
	\item espagnol (code ISO 639-1 ES).
\end{itemize}

\pagebreak

\section{Commandes, clés et options}

\subsection{Le plateau seul}

Le premier argument, \textit{optionnel}, entre \texttt{<...>} est la \Cle{langue} d'affichage peut-être choisie parmi :

\hfill\Cle{FR} (français, par défaut), \Cle{EN} (anglais), \Cle{DE} (allemand) et \Cle{ES} (espagnol).\hfill~

\smallskip

Le second argument, \textit{optionnel}, entre \texttt{[...]} propose les \Cle{clés} :

\begin{itemize}
	\item \Cle{Echelle} pour spécifier l'échelle d'affichage (l'unité de base étant 1~cm) ; \hfill~défaut : \Cle{1}
	\item \Cle{Echellelabels} pour spécifier l'échelle d'affichage des labels ; \hfill~défaut : \Cle{1}
	\item le booléen \Cle{Cadre} pour afficher un cadre autour du plateau ; \hfill~défaut : \Cle{true}
	\item le booléen \Cle{Labels} pour afficher le \textit{nom} des cases ;\hfill~défaut : \Cle{true}
	\item le booléen \Cle{Aide} pour afficher une aide pour repérer les cases ;\hfill~défaut : \Cle{false}
\end{itemize}

\begin{PresentationCode}{}
\PlateauScrabble[Labels=false,Echelle=0.55]\\     %plateau sans le nom des cases
\PlateauScrabble<EN>[Echelle=0.55,Cadre=false]    %plateau en anglais, sans cadre
\end{PresentationCode}

\newpage

\subsection{Le plateau avec des mots}

Il s'agit dans ce cas d'utiliser l'\textsf{environnement} ainsi que la \textsf{commande} spécifique pour placer les mots.

\smallskip

En ce qui concerne les options de l'\textsf{environnement}, ce sont les mêmes que pour la \textsf{commande} autonome.

\smallskip

Pour le placement des mots :

\begin{itemize}
	\item le premier argument, \textit{optionnel}, entre \texttt{[...]} est l'orientation du mot, à choisir entre \Cle{H} (par défaut) et \Cle{V} (en fait toute autre lettre que \Cle{H} !) ;
	\item le deuxième argument, \textit{obligatoire}, entre \texttt{\{...\}}, est le mot à placer, en majuscules ou minuscules ;
	\item le dernier argument, \textit{obligatoire}, entre \texttt{\{...\}}, correspond aux coordonnées de la case sur laquelle sera placée le début du mot (la case \texttt{(1,1)} étant la case au bord Bas/Gauche).
\end{itemize}

\textbf{Remarque 1 :} la langue choisie permettra également d'avoir les jetons avec le \textit{bon} nombre de points !

\smallskip

\textbf{Remarque 2 :} le \textit{blanc} (ou \textit{joker}) est obtenu par la lettre \texttt{*}.

\begin{PresentationCode}{}
\begin{EnvScrabbleFR}[Echelle=0.75,Labels=false,Aide]
	\ScrabblePlaceMot{TIKZ}{1,13}
	\ScrabblePlaceMot[V]{pstricks}{1,15}
	\ScrabblePlaceMot[V]{KaTeX}{3,13}
	\ScrabblePlaceMot{cleveref}{1,10}
	\ScrabblePlaceMot[V]{METAPOST}{7,11}
	\ScrabblePlaceMot{LUALATEX}{5,8}
	\ScrabblePlaceMot[V]{ProfLy*ee}{11,15}
	\ScrabblePlaceMot{PROFCOLLE*E}{1,1}
	\ScrabblePlaceMot{SYNTAXE}{7,5}
	\ScrabblePlaceMot[V]{STRIN*}{10,6}
\end{EnvScrabbleFR}
\end{PresentationCode}

\begin{PresentationCode}{}
\begin{EnvScrabbleFR}<FR>[Echelle=0.55]
	\ScrabblePlaceMot{TIKZ}{1,13}  \ScrabblePlaceMot[V]{pstricks}{1,15}
	\draw (7,15) node[font=\LARGE\sffamily] {Version française} ; %code rajouté
\end{EnvScrabbleFR}~~~
\begin{EnvScrabbleFR}<EN>[Echelle=0.55]
	\ScrabblePlaceMot{TIKZ}{1,13} \ScrabblePlaceMot[V]{pstricks}{1,15}
	\draw (7,15) node[font=\LARGE\sffamily] {Version anglaise} ; %code rajouté
\end{EnvScrabbleFR}\\
\begin{EnvScrabbleFR}<DE>[Echelle=0.55]
	\ScrabblePlaceMot{TIKZ}{1,13} \ScrabblePlaceMot[V]{pstricks}{1,15}
	\draw (7,15) node[font=\LARGE\sffamily] {Version allemande} ; %code rajouté
\end{EnvScrabbleFR}~~~
\begin{EnvScrabbleFR}<ES>[Echelle=0.55]
	\ScrabblePlaceMot{RO7OSO}{1,12} \ScrabblePlaceMot[V]{pstricks}{1,15}
	\draw (7,15) node[font=\LARGE\sffamily] {Version espagnole} ; %code rajouté
\end{EnvScrabbleFR}
\end{PresentationCode}

\newpage

\subsection{Mot \textit{en ligne}}

Il s'agit ici de proposer une commande pour insérer un mot en mode \textit{en ligne}, avec ajustement automatique de la taille et de la position.

\begin{PresentationCode}{listing only}
%commande pour placer un mot en ligne
\MotScrabble[clés]{mot}
\end{PresentationCode}

Le premier argument, \textit{optionnel}, entre \texttt{[...]} permet de paramétrer les \Cle{clés} :

\begin{itemize}
	\item \Cle{CouleurFond} pour la couleur des cases ; \hfill~défaut : \Cle{yellow!40}
	\item \Cle{Police} pour la police des caractères ; \hfill~défaut : \Cle{\textbackslash bfseries\textbackslash sffamily}
	\item \Cle{CouleurPolice} pour la couleur des caractères ; \hfill~défaut : \Cle{black}
	\item \Cle{Langue} pour choisir la langue (pour les nombres de points) ; \hfill~défaut : \Cle{FR}
	\item \Cle{Offset} pour spécifier un espacement horizontal entre les jetons ; \hfill~défaut : \Cle{0.1pt}
	\item \Cle{Echelle} pour spécifier une échelle de base pour les textes ; \hfill~défaut : \Cle{0.6}
	\item \Cle{Score} qui est un booléen pour afficher le score de chaque jeton.\hfill~défaut : \Cle{true}
\end{itemize}

\textbf{Remarque 1 :} le code se charge de positionner le jeton pour un alignement et une mise à l'échelle \textit{satisfaisants} en fonction de la police active.

\smallskip

\textbf{Remarque 2 :} le \textit{blanc} (ou \textit{joker}) est obtenu par le caractère \texttt{*}.

\begin{PresentationCode}{}
Essai de mot de Scrabble en ligne \MotScrabble{TE*ST} pour voir !
\end{PresentationCode}

\begin{PresentationCode}{}
{\Huge\sffamily En ligne \MotScrabble[CouleurFond=teal!5,CouleurPolice=orange]{PYTHAG*RE} pour voir !}
\end{PresentationCode}

\begin{PresentationCode}{}
\scalebox{3}[3]{En ligne \MotScrabble[Langue=EN,Police=\ttfamily,Echelle=0.75]{PYTHAG*RE} pour voir !}
\end{PresentationCode}

\begin{PresentationCode}{}
{\LARGE En ligne \MotScrabble[Score=false,Offset=1pt,CouleurFond=orange!50]{PSTRICKS} pour voir !}
\end{PresentationCode}

\newpage

\subsection{Commandes anglaises}

Le package ayant potentiellement une portée \textit{internationale}, les commandes existent également en version \textit{anglaise}. Dans ce cas, les \Cle{clés} sont également à donner en anglais.

\begin{PresentationCode}{}
\ScrabbleBoard[Scale=0.5] \ScrabbleBoard[Labels=false,Scale=0.5]
\end{PresentationCode}

\begin{PresentationCode}{}
\begin{EnvScrabble}[Scale=0.5]
	\ScrabblePutWord{TIKZ}{1,13}  \ScrabblePutWord[V]{pstricks}{1,15}
	\draw (7,15) node[font=\LARGE\sffamily] {Version anglaise} ; %code rajouté
\end{EnvScrabble}~~~
\begin{EnvScrabble}<ES>[Scale=0.5]
	\ScrabblePutWord{TIKZ}{1,13} \ScrabblePutWord[V]{pstricks}{1,15}
	\draw (7,15) node[font=\LARGE\sffamily] {Version espagnole} ; %code rajouté
\end{EnvScrabble}
\end{PresentationCode}

\newpage

\subsection{Lettres spéciales}

Il est possible d'utiliser des jetons spéciaux, pour les langues \textsf{<DE>} et \textsf{<ES>}, mais pour des raisons de compatibilité, les caractères spéciaux sont \textit{codés} par des chiffres :

\begin{itemize}
	\item \texttt{0} code la lettre \fbox{\"O} avec le score 8 ;
	\item \texttt{1} code la lettre \fbox{\"A} avec le score 6 ;
	\item \texttt{4} code la lettre \fbox{\"U} avec le score 6 ;
	\item \texttt{6} code la \textit{lettre} \fbox{CH} avec le score 5 ;
	\item \texttt{7} code la \textit{lettre} \fbox{\~{N}} avec le score 8 ;
	\item \texttt{8} code la \textit{lettre} \fbox{RR} avec le score 8 ;
	\item \texttt{9} code la \textit{lettre} \fbox{LL} avec le score 8.
\end{itemize}

\begin{PresentationCode}{}
{\Huge \MotScrabble[Langue=DE,Police=\ttfamily,Echelle=0.75]{104AAA}}

{\Huge \MotScrabble[Langue=ES,Police=\sffamily,Echelle=0.75]{RO7OSO689}}
\end{PresentationCode}

\newpage

\part*{Historique}

\verb|v0.1.6|~:~~~~Correction d'un bug avec les jokers ; améliorations du plateau \textsf{<DE>}

\verb|v0.1.5|~:~~~~Ajout de lettres spéciales pour \textsf{<DE>} et \textsf{<ES>}

\verb|v0.1.4|~:~~~~Ajout d'une commande pour placer en \textit{mot} en mode \og en ligne \fg

\verb|v0.1.3|~:~~~~Meilleure gestion de la saisie des mots (sans virgule, majuscule ou minuscule)

\verb|v0.1.2|~:~~~~Clé \textsf{<EchelleLabels>} pour modifier l'échelle des noms des cases

\verb|v0.1  |~:~~~~Version initiale

\end{document}